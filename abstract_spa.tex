\begin{center}
    \Large\bfseries Análisis evolutivo y estructural de las proteínas nsp10, 
    nsp14 y nsp16 para el diseño de una terapia antiviral pancoronavírica

\end{center} 

\vspace{0.5cm}

    \textbf {RESUMEN:} 

    Durante este siglo, se han registrado tres alertas sanitarias globales 
    causadas por la aparición de nuevos coronavirus zoonóticos que pueden 
    generar infecciones respiratorias agudas; SARS-CoV (2002), MERS-CoV (2012), 
    SARS-CoV-2 (2019). En respuesta a la posible aparición de futuros 
    coronavirus pandémicos, se destaca el desarrollo de múltiples terapias 
    antivirales que se dirijan a características víricas conservadas y que 
    puedan ser utilizadas en combinación, estableciéndose como la primera línea 
    de defensa terapéutica más efectiva. Dentro de este contexto, en este 
    Trabajo de Fin de Grado se realiza un análisis de la evolución y estructura 
    de dos proteínas clave en el complejo de replicación y transcripción de los 
    coronavirus, nsp14 y nsp16, así como del cofactor que ambas comparten, 
    nsp10. Las regiones conservadas de esta última podrían ser claves para el 
    diseño de pseudoligandos que ayuden a frenar el ciclo infectivo de 
    coronavirus en humanos.

\vspace{0.5cm}

    \textbf{Palabras clave:} coronavirus, evolución, estrategia antiviral, 
    filogenia, análisis estructural, nsp10, nsp14, nsp16.