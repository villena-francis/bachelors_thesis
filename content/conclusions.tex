\section{Conclusiones}

Los alineamientos de secuencias y las regiones seleccionadas según modelos
cristalográficos de las proteínas correspondientes proporcionan una visión 
acerca de la conservación de una posible diana contra la replicación de 
coronavirus humanos. En base a los resultados de este trabajo, podemos concluir 
lo siguiente:

\begin{enumerate}

\item Las nsp10/14/16 de la subfamilia \textit{Orthocoronavirinae} están bastante 
conservadas, especialmente las relativas a los géneros que reúnen a casi 
todos los coronavirus capaces de infectar a humanos: 
\textit{Alphacoronavirus} y \textit{Betacoronavirus}.

\item Según las estructuras experimentales de SARS-CoV-2 tomadas como 
referencia, la región de nsp10 con máxima interacción con nsp14-ExoN y nsp16
está constituida por 6 aminoácidos que configuran una cadena extendida sin 
estructura secundaria.

\item Las posiciones de la región de nsp10 citada en el apartado anterior 
están bastante conservadas dentro del conjunto de los géneros 
\textit{Alphacoronavirus}, \textit{Betacoronavirus} y 
\textit{Gammacoronavirus}. Sin embargo, \textit{Deltacoronavirus} solo 
preserva 3 posiciones hidrofóbicas comunes para toda la subfamilia 
\textit{Orthocoronavirinae}, concentrándose la mayor variabilidad de esta 
región en las posiciones hidrofílicas.

\end{enumerate}

Por todo ello, los análisis \textit{in silico} de este trabajo apoyan la 
idoneidad de la región de nsp10 con máxima interacción con nsp14-ExoN y 
nsp16 para su uso como molde en el diseño de posibles pseudoligandos con 
actividad antiviral pancoronavírica.