\begin{center}
    
\Large\bfseries Evolutionary and structural analysis of nsp10, nsp14 and 
nsp16 proteins for the design of a pancoronaviral antiviral therapy

\end{center} 

\vspace{0.5cm}

\textbf {ABSTRACT:} 

During this century, there have been three global health alerts caused by 
the emergence of new zoonotic coronaviruses that can cause acute respiratory
infections: SARS-CoV (2002), MERS-CoV (2012) and SARS-CoV-2 (2019). In 
response to the possible emergence of future pandemic coronaviruses, the 
development of multiple antiviral therapies that target conserved viral 
characteristics which can be used in combination is highlighted as the most 
effective first line of therapeutic defence. In this context, this Final 
Degree Project analyses the evolution and structure of two key proteins in 
the replication and transcription complex of coronaviruses, nsp14 and nsp16,
as well as the cofactor they share, nsp10. The conserved regions of the latter
could be key to the design of pseudoligands that help to stop the infectious
cycle of coronaviruses in humans.

\vspace{0.5cm}

\textbf{Keywords:} coronavirus, evolution, antiviral strategy, 
phylogeny, structural analysis, nsp10, nsp14, nsp16.