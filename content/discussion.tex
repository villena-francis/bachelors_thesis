\section{Discusión y futuras líneas}

\subsection{Análisis de los alineamientos de nsp10/14/16}

En cuanto a las secuencias de nsp10/14/16 de las especies de 
\textit{Orthocoronavirinae}, los alineamientos revelan en general gran 
número de posiciones altamente conservadas. Sin embargo, es importante tener
en cuenta que esta observación debe ser interpretada en el contexto de la 
amplia variabilidad genética de los virus en comparación con los organismos 
celulares. A simple vista los patrones observados no son tan canónicos como 
los que se pudiesen encontrar, por ejemplo, en alineamientos de proteínas 
implicadas en el desarrollo de vertebrados 
(\Cite{domingo_mutation_2021,duffy_why_2018,duffy_phylogenetic_2008,holmes_evolutionary_2009}).

La secuencia de la región seleccionada de nsp10, posiciones 45--52 del 
\href{https://htmlpreview.github.io/?https://github.com/villena-francis/bachelors_thesis/blob/main/data/align_no_ext_sp/nsp10/Trim_Alig_All_nsp10.html}{alineamiento de nsp10}, 
presenta una muy baja identidad de secuencia en 
\textit{Alphapironavirus bona} (\textit{Pitovirinae}) respecto a todas sus 
ortólogas de \textit{Orthocoronavirinae} con las que fue alineada. Además, 
para esta misma especie, en el 
\href{https://htmlpreview.github.io/?https://github.com/villena-francis/bachelors_thesis/blob/main/data/align_no_ext_sp/nsp14/Trim_Alig_All_nsp14.html}{alineamiento de nsp14}, 
la secuencia comienza a partir de la posición 42, lo cual evidencia la ausencia
de la región amino terminal que para el resto de coronavirus contiene todos 
los dominios de interacción con la región de nsp10 seleccionada para el 
objetivo principal de este trabajo, localizados a partir de la estructura la
estructura de referencia (\textbf{Figura 8}).

Las significativas disparidades observadas entre las secuencias de 
nsp10/14/16 de la especie perteneciente a la subfamilia \textit{Pitovirinae}
y las de coronavirus de aves y mamíferos (\textit{Orthocoronavirinae}) 
contrastan con el conocimiento existente sobre los dominios de interacción 
para estos últimos 
(\cite{baddock_characterization_2022,vithani_sars-cov-2_2021}). Se ha 
demostrado que mutaciones puntuales en residuos pertenecientes a la interfaz
de interacción de nsp14 y nsp10 producen una disminución de la fidelidad de 
la replicación viral, con el consiguiente aumento de la tasa de error y la 
inestabilidad genómica 
(\cite{bouvet_rna_2012,chen_biochemical_2011,ogando_curious_2019,rosas-lemus_crystal_2020,takada_genomic_2023}). 
A pesar de ello, las notables diferencias y huecos en las secuencias de 
nsp10 y nsp14 de \textit{Alphapironavirus bona} no parecen haber impedido 
que el tamaño de su genoma alcance las 36,6 kilobases 
(\cite{mordecai_endangered_2019}). Si estos hechos se mantuviesen para 
nuevas especies reconocidas de coronavirus externos a 
\textit{Orthocoronavirinae}, sería indicio de notables diferencias entre los 
modelos de interacción nsp10-nsp14 de coronavirus de aves y mamíferos 
respecto a los de otros vertebrados.

\subsection{Historia evolutiva de nsp10/14/16}

Como se indica en resultados, las secuencias relativas al virus del salmón 
del Pacífico fueron descartadas para la reconstrucción de las filogenias. 
Las distancias evolutivas inferidas para sus secuencias de nsp10/14/16 
vendrían dadas principalmente por la ausencia de posiciones comparables en 
esta candidata a especie externa. Además, esta falta de información 
aumentaría a su vez la probabilidad de posicionar a las secuencias del 
PsNV en otros nodos de las filogenias, reduciendo sus valores de
bootstrap y con ello la fiabilidad de las ramificaciones generadas.

Las relaciones filogenéticas de las proteínas nsp14 y nsp16 de los 
coronavirus que infectan aves y mamíferos (\textbf{Figuras 6} y \textbf{7}) 
concuerdan con las expectativas establecidas por la taxonomía filogenética 
asignada a sus respectivas especies (\cite{gulyaeva_nidovirus_2021,woo_family_2023,zhou_taxonomy_2021}). 
En ambos casos, se presume que las secuencias de ambas proteínas se 
diferenciaron desde el ancestro común de la subfamilia 
\textit{Orthocoronavirinae} en los linajes correspondientes a los cuatro 
géneros existentes.

La filogenia reconstruida utilizando las secuencias de nsp10 
(\textbf{Figura 5}) difiere en cierta medida de las relativas a nsp14 y 
nsp16. La agrupación de las secuencias de \textit{Alphacoronavirus} como un 
grupo monofilético dentro de \textit{Betacoronavirus} puede atribuirse a un 
resultado artefactual. Esto puede ser causado por la diferencia notable
en el número de residuos entre las proteínas nsp10/14/16, siendo el número 
de residuos en nsp10 aproximadamente cuatro y dos veces menor que los de 
nsp14 y nsp16, respectivamente. La reducción en el número de posiciones 
informativas y la excesiva conservación de las secuencias contribuyen a 
reducir de forma generalizada los valores de bootstrap de los nodos que 
sitúan a las secuencias de \textit{Alphacoronavirus} dentro de 
\textit{Betacoronavirus} en la filogenia obtenida para nsp10.

Estos hallazgos respaldarían la idoneidad de desarrollar un pseudoligando 
basado en este cofactor, ya que su conservación evolutiva le conferiría 
potencial para ser eficaz contra todos los coronavirus humanos actuales.

\subsection{Región de nsp10 seleccionada}

En base a los resultados obtenidos, la región de nsp10 con máxima 
interacción con nsp14-ExoN y nsp16 fue evidente; con un total de 6 residuos 
comunes en ambas interacciones. Además, se identificaron y añadieron otros 2
residuos contiguos adicionales que interactúan exclusivamente con nsp16 
(\textbf{Figura 8A}). Con relación al diseño de un pseudoligando, esta 
selección no solo sería relevante para una mayor selectividad de unión a 
dicha proteína diana, sino que también podría ayudar a reducir la 
probabilidad de agregaciones inespecíficas con otros componentes celulares.

Como se puede apreciar en los logos de secuencias (\textbf{Figura 9}), los 
residuos de la zona de máxima interacción de nsp10 que interactúan tanto con
nsp14-ExoN como con nsp16 (40--45) muestran una alta conservación en los 
géneros \textit{Alphacoronavirus}, \textit{Betacoronavirus} y 
\textit{Gammacoronavirus}. En esta misma región, los 
\textit{Deltacoronavirus} comparten con los géneros mencionados 
anteriormente solo tres posiciones con residuos hidrofóbicos (42, 44, 45). 
La preservación de estas tres posiciones en todo \textit{Orthocoronavirinae}
sugiere su papel esencial en la activación de las funciones de nsp14-ExoN y 
nsp16. En las estructuras presentadas como referencia (\textbf{Figura 8A} y
\textbf{8B}) puede observarse que dichas posiciones hidrofóbicas están 
involucradas en la formación de agrupaciones con otros residuos similares en
los dominios de interacción de estas enzimas, lo cual reduciría sus 
exposiciones al agua del citosol y haría estas uniones termodinámicamente 
favorables. Las interacciones del resto de posiciones de la región 
seleccionada (40, 41, 43, 46, 47) estarían mediadas por interacciones
polares. Sin embargo, para validar consistentemente esta última afirmación 
se requerirían análisis adicionales.

En el caso hipotético de utilizar un pseudoligando basado en los mismos 
residuos de un betacoronavirus, como el SARS-CoV-2, podría esperarse una 
efectividad similar contra la nsp14-ExoN de otros beta- y alfacoronavirus 
humanos debido a la alta identidad de secuencia que existe entre estos 
grupos. No obstante, y siguiendo con este ejemplo, su selectividad contra 
nsp16 sería más variable debido a las posiciones cambiantes del séptimo y 
octavo residuo del pseudoligando (47, 48), incluso dentro de los 
betacoronavirus.

Es importante mencionar que este estudio presenta limitaciones relacionadas
principalmente con la capacidad de computo del dispositivo doméstico 
utilizado. El objetivo principal de este trabajo es proporcionar indicios de
conservación evolutiva que puedan ser utilizados como estrategia antiviral. 
En este sentido, se justifica la necesidad de seguir profundizando en 
diversos aspectos, comenzando por el análisis cuantitativo de las energías 
de interacción entre los residuos de la región seleccionada y las proteínas 
diana correspondientes. Mediante estos análisis, se podría avanzar en
el diseño del pseudoligando al inferir los residuos más apropiados para las 
posiciones que en las zonas diana de nsp14-ExoN y nsp16 deberían establecer 
interacciones polares, asegurando su eficacia frente a un amplio espectro 
de coronavirus.